%%%%%%%%%%%%%%%%%%%%%%%%%%%%%%%%%%%%%%%%%%%%%%%%%%%%%%%%%%%%%%%%%%%%%%%%
%%%%%%%%%%%%%%%%%%%%%% Simple LaTeX CV Template %%%%%%%%%%%%%%%%%%%%%%%%
%%%%%%%%%%%%%%%%%%%%%%%%%%%%%%%%%%%%%%%%%%%%%%%%%%%%%%%%%%%%%%%%%%%%%%%%

%%%%%%%%%%%%%%%%%%%%%%%%%%%%%%%%%%%%%%%%%%%%%%%%%%%%%%%%%%%%%%%%%%%%%%%%
%% NOTE: If you find that it says                                     %%
%%                                                                    %%
%%                           1 of ??                                  %%
%%                                                                    %%
%% at the bottom of your first page, this means that the AUX file     %%
%% was not available when you ran LaTeX on this source. Simply RERUN  %%
%% LaTeX to get the ``??'' replaced with the number of the last page  %%
%% of the document. The AUX file will be generated on the first run   %%
%% of LaTeX and used on the second run to fill in all of the          %%
%% references.                                                        %%
%%%%%%%%%%%%%%%%%%%%%%%%%%%%%%%%%%%%%%%%%%%%%%%%%%%%%%%%%%%%%%%%%%%%%%%%

%%%%%%%%%%%%%%%%%%%%%%%%%%%% Document Setup %%%%%%%%%%%%%%%%%%%%%%%%%%%%

% Don't like 10pt? Try 11pt or 12pt
\documentclass[11pt]{article}

% The automated optical recognition software used to digitize resume
% information works best with fonts that do not have serifs. This
% command uses a sans serif font throughout. Uncomment both lines (or at
% least the second) to restore a Roman font (i.e., a font with serifs).
%\usepackage{times}
%\renewcommand{\familydefault}{\sfdefault}

% This is a helpful package that puts math inside length specifications
\usepackage{calc}
\usepackage{comment}
\frenchspacing

% Simpler bibsection for CV sections
% (thanks to natbib for inspiration)
\makeatletter
\newlength{\bibhang}
\setlength{\bibhang}{1em} %1em}
\newlength{\bibsep}
 {\@listi \global\bibsep\itemsep \global\advance\bibsep by\parsep}
\newenvironment{bibsection}%
        {\begin{enumerate}{}{%
%        {\begin{list}{}{%
       \setlength{\leftmargin}{\bibhang}%
       \setlength{\itemindent}{-\leftmargin}%
       \setlength{\itemsep}{\bibsep}%
       \setlength{\parsep}{\z@}%
        \setlength{\partopsep}{0pt}%
        \setlength{\topsep}{0pt}}}
        {\end{enumerate}\vspace{-.6\baselineskip}}
%        {\end{list}\vspace{-.6\baselineskip}}
\makeatother

% Layout: Puts the section titles on left side of page
\reversemarginpar

%
%         PAPER SIZE, PAGE NUMBER, AND DOCUMENT LAYOUT NOTES:
%
% The next \usepackage line changes the layout for CV style section
% headings as marginal notes. It also sets up the paper size as either
% letter or A4. By default, letter was used. If A4 paper is desired,
% comment out the letterpaper lines and uncomment the a4paper lines.
%
% As you can see, the margin widths and section title widths can be
% easily adjusted.
%
% ALSO: Notice that the includefoot option can be commented OUT in order
% to put the PAGE NUMBER *IN* the bottom margin. This will make the
% effective text area larger.
%
% IF YOU WISH TO REMOVE THE ``of LASTPAGE'' next to each page number,
% see the note about the +LP and -LP lines below. Comment out the +LP
% and uncomment the -LP.
%
% IF YOU WISH TO REMOVE PAGE NUMBERS, be sure that the includefoot line
% is uncommented and ALSO uncomment the \pagestyle{empty} a few lines
% below.
%

%% Use these lines for letter-sized paper
\usepackage[paper=letterpaper,
            %includefoot, % Uncomment to put page number above margin
            marginparwidth=1.2in,     % Length of section titles
            marginparsep=.05in,       % Space between titles and text
            margin=.8in,               % 1 inch margins
            includemp]{geometry}

%% Use these lines for A4-sized paper
%\usepackage[paper=a4paper,
%            %includefoot, % Uncomment to put page number above margin
%            marginparwidth=30.5mm,    % Length of section titles
%            marginparsep=1.5mm,       % Space between titles and text
%            margin=25mm,              % 25mm margins
%            includemp]{geometry}

%% More layout: Get rid of indenting throughout entire document
\setlength{\parindent}{0in}

\usepackage[shortlabels]{enumitem}

%% Reference the last page in the page number
%
% NOTE: comment the +LP line and uncomment the -LP line to have page
%       numbers without the ``of ##'' last page reference)
%
% NOTE: uncomment the \pagestyle{empty} line to get rid of all page
%       numbers (make sure includefoot is commented out above)
%
\usepackage{fancyhdr,lastpage}
\pagestyle{fancy}
%\pagestyle{empty}      % Uncomment this to get rid of page numbers
\fancyhf{}\renewcommand{\headrulewidth}{0pt}
\fancyfootoffset{\marginparsep+\marginparwidth}
\newlength{\footpageshift}
\setlength{\footpageshift}
          {0.5\textwidth+0.5\marginparsep+0.5\marginparwidth-2in}
%\lfoot{\hspace{\footpageshift}%
%       \parbox{4in}{\, \hfill %
%                    \arabic{page} %of \protect\pageref*{LastPage} % +LP
%                    \arabic{page}                               % -LP
%                    \hfill \,}}

% Finally, give us PDF bookmarks
\usepackage{color,hyperref}
\definecolor{maroon}{RGB}{100,0,0}
\hypersetup{colorlinks,breaklinks,
            linkcolor=maroon,urlcolor=maroon,
            anchorcolor=maroon,citecolor=maroon}

%%%%%%%%%%%%%%%%%%%%%%%% End Document Setup %%%%%%%%%%%%%%%%%%%%%%%%%%%%


%%%%%%%%%%%%%%%%%%%%%%%%%%% Helper Commands %%%%%%%%%%%%%%%%%%%%%%%%%%%%

% The title (name) with a horizontal rule under it
% (optional argument typesets an object right-justified across from name
%  as well)
%
% Usage: \makeheading{name}
%        OR
%        \makeheading[right_object]{name}
%
% Place at top of document. It should be the first thing.
% If ``right_object'' is provided in the square-braced optional
% argument, it will be right justified on the same line as ``name'' at
% the top of the CV. For example:
%
%       \makeheading[\emph{Curriculum vitae}]{Your Name}
%
% will put an emphasized ``Curriculum vitae'' at the top of the document
% as a title. Likewise, a picture could be included:
%
%   \makeheading[\includegraphics[height=1.5in]{my_picutre}]{Your Name}
%
% the picture will be flush right across from the name.
\newcommand{\makeheading}[2][]%
        {\hspace*{-\marginparsep minus \marginparwidth}%
         \begin{minipage}[t]{\textwidth+\marginparwidth+\marginparsep}%
             {\LARGE \bfseries #2 \hfill #1}\\[-0.3\baselineskip]%
                 \rule{\columnwidth}{.3pt}%
         \end{minipage}}

% The section headings
%
% Usage: \section{section name}
\renewcommand{\section}[1]{\pagebreak[3]%
    \hyphenpenalty=10000%
    \vspace{1.3\baselineskip}%
    \phantomsection\addcontentsline{toc}{section}{#1}%
    \noindent\llap{\scshape\smash{\parbox[t]{\marginparwidth}{\raggedright #1}}}%
    \vspace{-\baselineskip}\par}

% An itemize-style list with lots of space between items
\newenvironment{outerlist}[1][\enskip\textbullet]%
        {\begin{itemize}[#1,leftmargin=*]}{\end{itemize}%
         \vspace{-.6\baselineskip}}

% An environment IDENTICAL to outerlist that has better pre-list spacing
% when used as the first thing in a \section
\newenvironment{lonelist}[1][\enskip\textbullet]%
        {\begin{list}{#1}{%
        \setlength{\partopsep}{0pt}%
        \setlength{\topsep}{0pt}}}
        {\end{list}\vspace{-.6\baselineskip}}

% An itemize-style list with little space between items
\newenvironment{innerlist}[1][\enskip\textbullet]%
        {\begin{itemize}[#1,leftmargin=*,parsep=0pt,itemsep=0pt,topsep=0pt,partopsep=0pt]}
        {\end{itemize}}

% An environment IDENTICAL to innerlist that has better pre-list spacing
% when used as the first thing in a \section
\newenvironment{loneinnerlist}[1][\enskip\textbullet]%
        {\begin{itemize}[#1,leftmargin=*,parsep=0pt,itemsep=0pt,topsep=0pt,partopsep=0pt]}
        {\end{itemize}\vspace{-.6\baselineskip}}

% To add some paragraph space between lines.
% This also tells LaTeX to preferably break a page on one of these gaps
% if there is a needed pagebreak nearby.
\newcommand{\blankline}{\quad\pagebreak[3]}
\newcommand{\halfblankline}{\quad\vspace{-0.5\baselineskip}\pagebreak[3]}

% Uses hyperref to link DOI
\newcommand\doilink[1]{\href{http://dx.doi.org/#1}{#1}}
\newcommand\doi[1]{doi:\doilink{#1}}

% For \url{SOME_URL}, links SOME_URL to the url SOME_URL
\providecommand*\url[1]{\href{#1}{#1}}
% Same as above, but pretty-prints SOME_URL in teletype fixed-width font
\renewcommand*\url[1]{\href{#1}{\texttt{#1}}}

% For \email{ADDRESS}, links ADDRESS to the url mailto:ADDRESS
\providecommand*\email[1]{\href{mailto:#1}{#1}}
% Same as above, but pretty-prints ADDRESS in teletype fixed-width font
%\renewcommand*\email[1]{\href{mailto:#1}{\texttt{#1}}}

%\providecommand\BibTeX{{\rm B\kern-.05em{\sc i\kern-.025em b}\kern-.08em
%    T\kern-.1667em\lower.7ex\hbox{E}\kern-.125emX}}
%\providecommand\BibTeX{{\rm B\kern-.05em{\sc i\kern-.025em b}\kern-.08em
%    \TeX}}
\providecommand\BibTeX{{B\kern-.05em{\sc i\kern-.025em b}\kern-.08em
    \TeX}}
\providecommand\Matlab{\textsc{Matlab}}

%%%%%%%%%%%%%%%%%%%%%%%% End Helper Commands %%%%%%%%%%%%%%%%%%%%%%%%%%%

%%%%%%%%%%%%%%%%%%%%%%%%% Begin CV Document %%%%%%%%%%%%%%%%%%%%%%%%%%%%

\begin{document}
\makeheading{Kareem Haggag \hfill{\footnotesize{Last updated on May 20, 2018}}}

\section{Contact Information}

% NOTE: Mind where the & separators and \\ breaks are in the following
%       table.
%
% ALSO: \rcollength is the width of the right column of the table
%       (adjust it to your liking; default is 1.85in).
%
\newlength{\rcollength}\setlength{\rcollength}{2.47in}%
%
\begin{tabular}[t]{@{}p{\textwidth-\rcollength}p{\rcollength}}
Carnegie Mellon University  & (773) 710-0773 \\
Social and Decision Sciences & \email{kareem.haggag@cmu.edu}\\
Porter Hall, 208-H   & \href{http://www.kareemhaggag.com/}{http://www.kareemhaggag.com} \\
Pittsburgh, PA 15213  & Citizenship: United States \\
\end{tabular}

%\section{Objective}

%Insert text here if you want to
%\begin{innerlist}
% Blah
%\end{innerlist}
\section{Academic Positions}

\textbf{Carnegie Mellon University}, Pittsburgh, PA \hfill{January 2017 - } 
\begin{outerlist}
\vspace{-.05in}
\item[] Social and Decision Sciences Department
\vspace{-.1in}
\item[] Assistant Professor
\end{outerlist}
\vspace{.15in}

\textbf{Yale University}, New Haven, CT \hfill{July 2016 - December 2016}
\begin{outerlist}
\vspace{-.05in}
\item[] Innovations for Poverty Action - Financial Inclusion Program
\vspace{-.1in}
\item[] Postdoctoral Fellow
\end{outerlist}
\vspace{-.1in}

\section{Education}

\textbf{University of Chicago, Booth School of Business}, Chicago, IL
\begin{outerlist}
\vspace{-.05in}
\item[] Ph.D. in Economics \hfill{2010 - 2016}
\end{outerlist}
\vspace{.1in}

\textbf{University of Tennessee}, Knoxville, TN
\begin{outerlist}
\vspace{-.05in}
\item[] B.A. in Economics \hfill{2004 - 2008}
\end{outerlist}
\vspace{-.1in}

\section{Research Interests}
\vspace{0.05in}
Behavioral Economics, Applied Microeconomics
\vspace{0.05in}

\section{Publications}

\href{http://www.kareemhaggag.com/f/Learning_by_Driving.pdf}
{\textbf{Learning By Driving: Productivity Improvements by New York City Taxi}} \href{https://home.uchicago.edu/~haggag/Learning_by_Driving.pdf}{\textbf{Drivers}} (with Brian McManus \& Giovanni Paci) \\
\textbf{American Economic Journal: Applied Economics}, 2017, 9(1): 70-95. \\
\vspace{-.1in} \\
\textit{Abstract}: We study learning by doing (LBD) by New York City taxi drivers, who have substantial discretion over their driving strategies and receive compensation closely tied to their success in finding customers. In addition to documenting significant learning by these entrepreneurial agents, we exploit our data's breadth to investigate the factors that contribute to driver improvement across a variety of situations. New drivers lag farther behind experienced drivers when in difficult situations. Drivers benefit from accumulating neighborhood-specific experience, which affects how they search for their next customers. \\

\href{https://www.kareemhaggag.com/f/Default_Tips.pdf}{\textbf{Default Tips}} (with Giovanni Paci) \\
\textbf{American Economic Journal: Applied Economics}, 2014, 6(3): 1-19 \\
\vspace{-.1in} \\
\textit{Abstract}: We examine the role of defaults in high-frequency, small-scale choices using unique data on over 13 million NYC taxi rides. We exploit a shift in the set of default tip suggestions presented to customers prior to payment, as the base fare changes from below \$15 to above \$15. Using a regression discontinuity design, we show that default suggestions have a large impact on tip amounts. These results are supported by a secondary analysis that uses the quasi-random assignment of customers to different cars to examine default effects on all fares above \$15. Finally, we highlight a potential cost of setting defaults too high, as a higher proportion of customers opt to leave no credit card tip when presented with the higher suggested amounts.
\vspace{.07in} \\
Editors' Choice: \href{http://www.sciencemag.org/content/345/6203/twil.full}{Science Magazine, Vol 345(6203)} \\

\section{Working Papers}

\href{https://www.kareemhaggag.com/f/Attribution_Bias.pdf}{\textbf{Attribution Bias in Consumer Choice}} (with Devin G. Pope, Kinsey B. Bryant-Lees, and Maarten W. Bos) \\
Conditionally accepted at \textbf{Review of Economic Studies} \\
\vspace{-.1in} \\
\textit{Abstract}: When judging the value of a good, people may be overly influenced by the state in which they previously consumed it. For example, someone who tries out a new restaurant while very hungry may subsequently rate it as high quality, even if the food is mediocre. We produce a simple framework for this form of attribution bias that embeds a standard model of decision making as a special case. We test for attribution bias across two consumer decisions. First, we conduct an experiment in which we randomly manipulate the thirst of participants prior to consuming a new drink. Second, using data from thousands of amusement park visitors, we explore how pleasant weather during their most recent trip affects their stated and actual likelihood of returning. In both of these domains, we find evidence that people misattribute the influence of a temporary state to a stable quality of the consumption good. We provide evidence against several alternative accounts for our findings and discuss the broader implications of attribution bias in economic decision making.
\vspace{-.05in} \\

\href{https://www.kareemhaggag.com/f/Attribution_Bias_USMA.pdf}{\textbf{Attribution Bias in Major Decisions: Evidence from the United States Military Academy}} (with Richard W. Patterson, Nolan G. Pope, and Aaron Feudo) \\
\vspace{-.1in} \\
\textit{Abstract}: Using administrative data, we study the role of attribution bias in a high-stakes, consequential decision: the choice of a college major. Specifically, we examine the influence of fatigue experienced during exposure to a general education course on whether students choose the major corresponding to that course. To do so, we exploit the conditional random assignment of student course schedules at the United States Military Academy. We find that students who are assigned to an early morning (7:30 AM) section of a general education course are roughly 10\% less likely to major in that subject, relative to students assigned to a later time slot for the course. We find similar effects for fatigue generated by having one or more back-to-back courses immediately prior to a general education course that starts later in the day. Finally, we demonstrate that the pattern of results is consistent with attribution bias and difficult to reconcile with competing explanations.
\vspace{-.05in}

\section{Work in Progress}
{\textbf{Inaccurate Statistical Discrimination}} (with J. Aislinn Bohren, Alex Imas, and Devin G. Pope) \\
\vspace{-.35in} \\

\section{Work Experience}
\textbf{Innovations for Poverty Action} \hfill {June 2008 - July 2010} \\
Project Associate

\section{Professional Activities}
Russell Sage Foundation Summer Institute in Behavioral Economics (2014) \\
Referee Service: \textit{American Economic Journal: Applied Economics}, \textit{American Economic Review}, \textit{Econometrica}, \textit{Economic Journal}, \textit{Journal of Behavioral and Experimental \\ Economics}, \textit{Journal of Economic Behavior and Organization}, \textit{Labour Economics}, \\ \textit{Management Science}, \textit{Review of Economics and Statistics}

\section{Invited \\ Presentations}
\textbf{2017}: University of Pittsburgh, Yale University, Penn-CMU Roybal Center Retreat \\

\textbf{2016}: Carnegie Mellon University, Cornell University, 
Early-Career Behavioral \\ Economics Conference (Bonn, Germany), SITE Experimental Conference (Stanford), University of California - San Diego (Rady), University of Oregon, University of Tennessee \\

\textbf{2013}: 9th Annual Whitebox Advisors Graduate Student Conference (Yale SOM) 

\section{Fellowships}
Agency of Healthcare Research and Quality T32 Training Fellowship \hfill{2015-2016} \\
Russell Sage Foundation Small Grant in Behavioral Economics \hfill{2013} \\
University of Chicago Booth Ph.D. Fellowship \hfill{2010-2015} \\

\section{References}

\begin{tabular}[t]{@{}p{\textwidth-\rcollength}p{\rcollength}}
Devin Pope  & John List \\
Booth School of Business & Department of Economics \\
University of Chicago & University of Chicago \\
devin.pope@chicagobooth.edu & jlist@uchicago.edu \\
\\
Marianne Bertrand  & Richard Thaler \\
Booth School of Business & Booth School of Business \\
University of Chicago & University of Chicago \\
marianne.bertrand@chicagobooth.edu & richard.thaler@chicagobooth.edu \\
\end{tabular}


\end{document}